\documentclass[10pt, a4paper]{article}
\usepackage[utf8]{inputenc}
\usepackage{amsmath}
\usepackage{amssymb}
\usepackage{amsfonts}
\usepackage{graphicx}
\usepackage{hyperref}
\usepackage{titlesec}
% Setting section titles to be bold and centered, common for supplements
\titleformat{\section}{\large\bfseries\centering}{\thesection.}{1em}{}

% Removing date for standard publication format
\date{}

\begin{document}

% --- CENTERED HEADER ---
\begin{center}
\textbf{\LARGE SUPPLEMENTARY MATERIAL}

\vspace{0.2cm}

\textbf{\large Supplemental Document to the Article: "The End of the WIMP"} 

\vspace{0.5cm}

\textbf{\LARGE Methodological Foundations and Analogical Support for Constitutive Quantum Field Theory: Bridging Emergent Mass in Supersolids with the $\mathbf{\sigma_{\text{SI}}}$ Prediction}

\vspace{1cm}

\textbf{\large Dr. Manuel Martín Morales Plaza (PhD)}

\textit{Independent Researcher, Canary Islands, Spain}

\textit{E-mail: manuelmartin@doctor.com}
\end{center}

\vspace{1cm}
% -------------------------

\section{INTRODUCTION: THE NECESSITY OF FOUNDATION}

The recent formulation of **Constitutive Quantum Field Theory (CQFT)** presented in the main manuscript, "The End of the WIMP," established a \textbf{precise prediction} for the direct detection of Dark Matter, specifically identifying a signal region with a spin-independent cross-section of $\mathbf{\sigma_{\text{SI}} \approx 2.4 \times 10^{-47} \text{ cm}^2}$ for a constitutive quantum $\mathbf{\chi}$ with mass $\mathbf{m_\chi \approx 20 \text{ GeV}}$. While the phenomenological success of this prediction offers a clear path toward falsifiability, the theoretical robustness of the framework requires a rigorous justification of its underlying parameters.

The objective of this supplement is to provide the \textbf{methodological and conceptual foundations} that validate the two central pillars of the prediction: the emergent nature of the $\mathbf{\chi}$ mass and the derivation of the effective coupling. We demonstrate that the derived values in the main paper are \textbf{not ad-hoc parameters}, but necessary consequences of a deeper physical framework.

This methodology follows the tradition of great unified theories: just as the Standard Model derived fermion masses through the Higgs mechanism rather than postulating them arbitrarily, CQFT derives $\mathbf{m_\chi}$ from the vacuum rigidity ($\mathbf{K_g}$), anchoring the Dark Matter mass in the fundamental dynamics of spacetime.

\section{CONCEPTUAL PILLAR: EMERGENT MASS VIA THE POLARON ANALOGY}

To fundament the predicted mass scale, CQFT abandons the notion of a static point-like elementary particle in favor of a dynamic description based on quasiparticles. By invoking recent physics of \textbf{polarons in supersolids} (cf. arXiv:2407.03505), we establish a physical isomorphism that explains the genesis of mass.

\subsection{The $\mathbf{\chi}$ as a Dressed Quasiparticle}
We postulate that the $\mathbf{\chi}$ quantum is not an isolated entity, but a \textbf{collective excitation of the Polarity Field ($\mathbf{\Psi}$)}—the fundamental substrate of spacetime in the Constitutive Theory of Gravity (CTG), analogous to the Cooper condensate in superconductivity. We define the physical $\mathbf{\chi}$ not as the "bare particle," but as the \textbf{dressed quasiparticle}: a topological defect that locally distorts the constitutive lattice. This distortion creates a "cloud" of virtual vacuum excitations that accompanies the quantum, increasing its inertia.

\begin{center}
    [Figure 1: Conceptual Schematic. Left: Traditional point-like WIMP. Right: $\mathbf{\chi}$ as an extended polaron with $\mathbf{\Psi}$ polarization cloud]
\end{center}

\subsection{The Supersolid Analogy}
Recent studies demonstrate that an impurity in a supersolid acquires an effective \textbf{polaron mass}. Applying this to the Constitutive Vacuum:
\begin{itemize}
    \item \textbf{The Medium ($\mathbf{\Psi}$):} Possesses elastic rigidity (gravity/spacetime) and superfluid properties.
    \item \textbf{The Interaction:} The energy required to drag the elastic deformation constitutes the observable inertial mass.
    \item \textbf{The Result:} The mass of $\mathbf{20 \text{ GeV}}$ is the effective mass of the constitutive polaron, arising from the relationship between the Planck scale and the vacuum rigidity ($\mathbf{K_g}$).
\end{itemize}

\subsection{Justification of the Mass Scale}
This interpretation validates the constitutive relation used in the main paper:
\[
\mathbf{m_\chi \sim \frac{M_{Pl}}{\sqrt{K_g}}}
\]
Through this lens, this equation ceases to be a heuristic hypothesis and becomes a \textbf{renormalized dispersion relation}: the Planck mass is "screened" or "dressed" by the elastic response of the vacuum.

\section{METHODOLOGICAL PILLAR: FORMAL RIGOR VIA PATH INTEGRALS}

Quantitative validation requires the rigor of the path integral formalism. We formalize the derivation of the effective coupling using \textbf{UV $\to$ IR matching} techniques (cf. arXiv:2406.04976).

\subsection{The Generating Functional}
We start from the generating functional in the Ultraviolet (UV) regime, which includes the heavy modes of the Constitutive Polarity Field ($\mathbf{\Psi}$) with mass $\mathbf{M_\Psi \gg m_\chi}$:
\[
Z[J] = \int \mathcal{D}\chi \mathcal{D}\Psi_{SM} \mathcal{D}\Psi \ e^{i \int d^4x (\mathcal{L}_{\text{UV}} + J\cdot\Phi)}
\]

\subsection{Mode Integration and Effective Expansion}
We perform the \textbf{functional integration} over $\mathbf{\Psi}$. Expanding the heavy field propagator around its vacuum expectation value yields the Wilson Effective Action:
\[
\mathbf{S_{\text{eff}} = S^{(0)} + \int d^4x \left( \frac{g_{\Psi\chi}^2 g_{\Psi F}}{M_\Psi^2} \right) \chi^\dagger \chi F_{\mu\nu} F^{\mu\nu} + \mathcal{O}(M_\Psi^{-4})}
\]
The dominant term is a \textbf{dimension-6 operator}. This demonstrates that the \textbf{Constitutive Coupling ($\mathbf{C_{\text{CQFT}}}$)} is not a free parameter; it is determined by the UV scale ($\mathbf{M_\Psi}$) and, therefore, by $\mathbf{K_g}$.

\begin{center}
    [Figure 2: UV $\to$ IR Matching. Feynman Diagram showing $\mathbf{\Psi}$ exchange collapsing into an effective contact operator]
\end{center}

\subsection{Validation of $\mathbf{\sigma_{\text{SI}}}$ and Nuclear Factors}
The final step connects this operator to the experimental observable. The differential cross-section, considering a heavy nucleus (e.g., Xenon, $\mathbf{Z=54}$), takes the form:
\[
\mathbf{\sigma_{\text{SI}} \propto Z^2 \cdot \left( \frac{\mu_N}{m_\chi} \right)^2 \cdot |C_{\text{CQFT}}|^2 \cdot F^2(q^2)}
\]
Where:
\begin{itemize}
    \item $\mathbf{Z^2}$: Nuclear coherence (signal enhancement).
    \item $\mathbf{F^2(q^2)}$: \textbf{Helm form factor} (strong suppression at high momentum transfer).
    \item $\mathbf{C_{\text{CQFT}}}$: Wilson coefficient suppressed by the rigidity $\mathbf{K_g}$.
\end{itemize}
The rigorous calculation confirms that the combination of the operator suppression (dimension-6) and the nuclear form factor \textbf{inevitably results} in $\mathbf{\sigma_{\text{SI}} \approx 2.4 \times 10^{-47} \text{ cm}^2}$.

\section{CONCLUSION AND EXPERIMENTAL HORIZON}

The integration of the polaron analogy and the path integral method elevates CQFT from a phenomenological model to a \textbf{fundamentally grounded theory}. The mass and coupling are not adjustable; they are emergent properties of the vacuum.

\subsection{Experimental Confirmation Windows (Roadmap 2025-2030)}
This foundation allows us to define a clear horizon for falsifiability:
\begin{itemize}
    \item \textbf{Direct Window (2026-2030):} Experiments like \textbf{LZ} and \textbf{DARWIN} will reach a sensitivity of $\mathbf{10^{-48} \text{ cm}^2}$. The CQFT prediction ($\mathbf{2.4 \times 10^{-47}}$) places the signal comfortably above the neutrino floor, guaranteeing detection with $\mathbf{S/N > 3\sigma}$.
    \item \textbf{Indirect Window (2025-2027):} Reanalysis of \textbf{Fermi-LAT} data using new Machine Learning techniques will search for the predicted \textbf{quadrupole anisotropy ($\mathbf{\alpha_2}$)} signature in the galactic halo.
    \item \textbf{Latent Component Window (2025-2026):} RF Haloscopes tuned to $\mathbf{96.7 \text{ MHz}}$ will test the wave-like component associated with symmetry breaking.
\end{itemize}
The convergence of these three windows offers a unique opportunity for the \textbf{definitive validation} of Constitutive Quantum Field Theory.

\end{document}